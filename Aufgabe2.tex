\documentclass{article}
\author{Florine Winter}
\title{CS102 \LaTeX \"Ubung}
\date{26. Oktober 2014}
\begin{document}
\maketitle
\section{Das ist mein erster Abschnitt}
Das ist mein eigener Text.
\section{Tabelle}
Das ist die Tabelle.\\

\begin{tabular}{c|c|c|c}

 & Punkte erhalten & Punkte m\"oglich & \% \\
\hline
Aufgabe1 & 2 & 4 & 0.5 \\
Aufgabe2 & 3 & 3 & 1 \\
Aufgabe3 & 3 & 3 & 1 \\

\end{tabular}\\
\begin{center}
Diese Tabelle beinhaltet dieselben Werte.
\end{center}
\section{Ich war hier}
Nathalia M\"unch
\section{Formeln}
\subsection{Pythagoras}
Der Satz des Pythagoras errechnet sich wie folgt:
$a^2 + b^2 = c^2$. Daraus k\"onnen wir die L\"ange der Hypothenuse wie folg berechnen: $c= \sqrt{a^2 + b^2}$
\subsection{Summen}
Wir k\"onnen auch die Formel f\"ur eine Summe angeben:\\
\\
\centering
$s= \sum\limits_{i=1}^{n}i=\frac{n*(n+1)}{2}$ (1)
\end{document}
